% !TEX TS-program = xelatex
% !TEX encoding = UTF-8 Unicode
% !Mode:: "TeX:UTF-8"

\documentclass{resume}
% usepackage{zh_CN-Adobefonts_external} % Simplified Chinese Support using external fonts (./fonts/zh_CN-Adobe/)
% \usepackage{NotoSansSC_external}
% \usepackage{NotoSerifCJKsc_external}
% \usepackage{zh_CN-Adobefonts_internal} % Simplified Chinese Support using system fonts
\usepackage{linespacing_fix} % disable extra space before next section
\usepackage{cite}

\begin{document}
\pagenumbering{gobble} % suppress displaying page number

\name{Wanru Zhao}

\basicInfo{
  \email{zhaowrenee@gmail.com} \textperiodcentered\ 
  \phone{(+86) 188-100-10893} \textperiodcentered\
  \faicon{location-arrow}北京市海淀区学院路37 \textperiodcentered\
  \github[Ryan0v0]{https://github.com/Ryan0v0}}
 
\section{\faGraduationCap\ Education}
\datedsubsection{\textbf{北京航空航天大学}, 北京, 中国}{2017 年9 月-- 至今}
\textit{本科生}\ 计算机科学与技术, 预计 2021 年 7 月毕业
\begin{itemize}
  \item 专业: 计算机科学与技术
  \item GPA: 3.6/4.0	 排名:36/244
%  \item 主修课程: 面向对象设计与构造 97, 离散数学 98, C++与C#程序设计 97
\end{itemize}
\datedsubsection{\textbf{剑桥大学暑期学校}, 剑桥, 英国}{2018 年7 月 -- 2018 年8 月}
\begin{itemize}
\item 唐宁学院 (Downing College)
  \item 双A 成绩结业
\end{itemize}

\section{\faUsers\ 研究经历}
\datedsubsection{\textbf{软件开发环境国家重点实验室}, 北航}{2019 年7 月 -- 至今}
\begin{onehalfspacing}
\role{本科实习生}{参与论文}
{\emph{Graph Attention based Proposal 3D ConvNets for Action Detection,}\textbf{AAAI 2020, submitted}}\\
{Jun Li, Xianglong Liu, Zhuofan Zong, \textbf{Wanru Zhao}, Minyuan Zhang, Jingkuan Song}
\end{onehalfspacing}

\datedsubsection{\textbf{群体智能实验室}, 北航}{2019 年4 月 -- 至今}
\begin{onehalfspacing}
\role{本科实习生}{导师: 童咏昕}
研究方向: 大数据与群体智能
\begin{itemize}
  \item 与滴滴公司合作, 网约车动态定价机制设计
  \item 空间众包与时空数据的处理分析
  \item 强化学习与联邦学习
\end{itemize}
\end{onehalfspacing}

\section{\faUsers\ 项目经历}

\datedsubsection{\textbf{基于GAN的手写汉字字体大规模生成系统}}{2019 年5 月 -- 至今}
\role{2019年第十三届国家大学生创新创业计划项目}{小组成员}
\begin{onehalfspacing}
网页前端开发,算法改进
\begin{itemize}
  \item 借助图像翻译的原理将字型规范且字集完整的标准字体通过生成式对抗网络(GAN)的方法翻译为具有其它风格特征的新字体
  \item 用于交互的相关网站和APP
\end{itemize}
\end{onehalfspacing}

\datedsubsection{\textbf{基于树莓派的三维人脸识别系统}}{2018 年12 月 -- 2019 年3 月}
\role{北京航空航天大学"冯如杯"学生学术科技作品竞赛作品}{团队成员}
\begin{onehalfspacing}
硬件环境搭建,论文撰写
\begin{itemize}
  \item 采用Raspberry Pi 3 作为硬件计算平台
  \item 在OpenCV3 环境中采用Caffe 框架载入预训练神经网络 
  \item 基于SGBM 算法的双目视觉三维重建
\end{itemize}
\end{onehalfspacing}

\datedsubsection{\textbf{卢瑟福原子结构动画模型制作与静态模型渲染}}{2019 年7 月 -- 2019 年8 月}
\role{C++}{个人项目}
\begin{onehalfspacing}
计算机图形学, 《计算机动画算法与技术》课程优秀大作业
\begin{itemize}
  \item OpenGL 库实现类太阳系的卢瑟福原子动态三维球体模型, 按照原比例和运行速率仿真,通过粒子系统、纹理、光照贴图等达到动态逼真效果, 可通过鼠标与键盘操纵摄像机的角度与远近
  \item 基于smallpt实现卢瑟福原子模型的静态三维造型与渲染,实现光线追踪、纹理映射、景深等效果
  \item OpenMP 并行及GPU 加速
\end{itemize}
\end{onehalfspacing}

\datedsubsection{\textbf{多电梯调度系统}}{2019 年4 月 -- 2019 年5 月}
\role{JAVA}{个人项目}
\begin{onehalfspacing}
面向对象程序设计
\begin{itemize}
  \item 使用JAVA多线程技术模拟实际写字楼不同功能的多电梯之间和乘客请求的交互
  \item 在鲁棒性测试和性能测试中表现良好
  \item 项目工程化,有比较完整的代码注视,流程图,类图,时序图和README文档
\end{itemize}
\end{onehalfspacing}

\datedsubsection{\textbf{JAVA小游戏软件开发}}{2018 年9 月 -- 2018 年12 月}
\role{JAVA}{个人项目}
\begin{onehalfspacing}
基于Java的游戏:吞食鱼、飞机大战
\begin{itemize}
  \item 模拟实现相应游戏场景和规则, 分为开始界面、难度选择界面、游戏界面、装备商城、游戏排行榜, 具有完整的游戏规则和多种游戏道具, 可通过玩家每轮游戏积分自动调整游戏难度 
  \item 使用AWT 实现图形界面,多线程实现不同对战关卡
  \item 文档基于软件工程标准,除去界面程序外项目主体代码量约3500行
\end{itemize}
\end{onehalfspacing}

\datedsubsection{\textbf{Unity3D游戏开发}}{2018 年11 月 -- 2018 年12 月}
\role{C\#}{团队项目, 线上投票第四名}


% Reference Test
%\datedsubsection{\textbf{Paper Title\cite{zaharia2012resilient}}}{May. 2015}
%An xxx optimized for xxx\cite{verma2015large}
%\begin{itemize}
%  \item main contribution
%\end{itemize}

\section{\faCogs\ IT 技能}
% increase linespacing [parsep=0.5ex]
\begin{itemize}[parsep=0.5ex]
  \item 编程语言: C/C++, Java, Python, SQL, Linux Shell, Verilog HDL, Mips汇编,  \LaTeX
  \item 平台: Windows, Linux, OS X
  \item 软件: Matlab, SPSS
\end{itemize}

\section{\faHeartO\ 获奖情况}
\begin{onehalfspacing}
\datedline{\textit{铜奖}, ICPC国际大学生程序设计竞赛中国邀请赛(南昌)暨国际丝绸之路程序设计竞赛}{2019 年6 月}
\datedline{\textit{银奖}, CCPC中国大学生程序设计竞赛-女生专场(WFinal)}{2018 年6 月}
\datedline{\textit{第2名},北京林业大学“华为杯”程序设计竞赛暨高校邀请赛}{2019 年6 月}
\datedline{\textit{第15名},第十四届"商汤杯"北京航空航天大学程序设计竞赛决赛}{2019 年4 月}
\datedline{\textit{三等奖},第二十八届“冯如杯”学生创意论文大赛}{2018 年3 月}
\datedline{\textit{铜牌},CCF NOI 全国青少年信息学奥林匹克竞赛}{2016 年7 月}
\datedline{\textit{一等奖}, CCF NOIp 全国青少年信息学奥林匹克联赛}{2015年11月}
\datedline{北京航空航天大学计算机学院三好学生}{2018 年6 月}
\datedline{北京航空航天大学校级优秀奖学金:学科竞赛奖学金一等奖}{2018 年6 月}
\end{onehalfspacing}

\section{\faUsers\ 校园经历}
\begin{itemize}[parsep=0.5ex]
	\item 北京航空航天大学微软学生俱乐部(MSC)技术部核心成员
%\datedline{北京航空航天大学微软学生俱乐部(MSC)技术部核心成员}{2018 年9 月 -- 至今}
	\item 北京航空航天大学学生会创新创业中心干事
%\datedline{北京航空航天大学学生会创新创业中心干事}{2017 年10 月 -- 2018年7 月}
	\item 热爱户外运动, 2018 年夏攀登四川贡嘎雪山卫峰那玛峰(海拔 5588m) 
%\datedline{热爱户外运动, 攀登四川贡嘎雪山卫峰那玛峰(海拔 5588m) }
	\item 热爱音乐与美术,葫芦丝九级,选修《西方美术与鉴赏》通识课程获得满分
%\datedline{葫芦丝 九级}
\end{itemize}
\end{document}
